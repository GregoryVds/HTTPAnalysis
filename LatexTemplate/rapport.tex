\documentclass[a4paper,11pt,final]{article}
% Pour une impression recto verso, utilisez plutôt ce documentclass :
%\documentclass[a4paper,11pt,twoside,final]{article}

\usepackage[english,francais]{babel}
\usepackage[utf8]{inputenc}
\usepackage[T1]{fontenc}
\usepackage[pdftex]{graphicx}
\usepackage{setspace}
\usepackage{hyperref}
\usepackage[french]{varioref}

\newcommand{\reporttitle}{LINGI1341: Réseaux informatiques}     % Titre
\newcommand{\reportauthor}{Grégory \textsc{Vander Schueren}} % Auteur
\newcommand{\reportsubject}{HTTP Analysis} % Sujet
\newcommand{\reportdate}{13 Novembre 2014} % Sujet
\newcommand{\HRule}{\rule{\linewidth}{0.5mm}}
\setlength{\parskip}{1ex} % Espace entre les paragraphes

\hypersetup{
    pdftitle={\reporttitle},%
    pdfauthor={\reportauthor},%
    pdfsubject={\reportsubject},%
    pdfkeywords={rapport} {vos} {mots} {clés}
}

\begin{document}
  % Inspiré de http://en.wikibooks.org/wiki/LaTeX/Title_Creation

\begin{titlepage}

\begin{center}


\textsc{\LARGE Université catholique de Louvain}

\vspace{4.0cm}

\begin{spacing}{2.5}
\textsc{\Large \reportsubject}\\[0.5cm]
{\huge \bfseries \reporttitle}\\[0.4cm]
 \end{spacing}

\begin{minipage}[t]{0.5\textwidth}
  \begin{flushleft} \large
    
  \end{flushleft}
\end{minipage}

\vfill
\reportauthor \\
\reportdate

\end{center}

\end{titlepage}

  \cleardoublepage

  \section{Présentation du site internet analysé}

  Le site internet analysé (techcrunch.com) est un site d'information spécialise dans les entreprises actives dans le domaine l'internet, aussi bien startups qu'entreprises côtées en bourse, les nouveaux produits internets et les nouvelles du monde technologique en général.

  Fondé en juin 2005 par Michael Arrington, Techcrunch comptabilise 37 million de pages vues par mois. Racheté en septembre 2010 par AOL, Techcrunch compte aujourd'hui plus de 40 employés et ses bureaux sont situés à San Fransisco.\footnote{http://techcrunch.com/about/\#about-tc} Le montant du deal de rachat par AOL n'a pas été communiqué publiquement mais est estimé à \$40 millions. \footnote{http://www.businessinsider.com/aol-techcrunch-price-25-million-2010-9}


\end{document}